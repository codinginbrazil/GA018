\subsubsection{Resultados do Método de Newton}
O Método de Newton aplicado a Equação~\ref{eq:2}, tendo a precisão de $10^{-8}$ e ponto de partida o ponto $x_{0} = (1,1,1,1)$ o resultado está disponível na Tabela~\ref{tab:sys-nr}, no qual representa todas as iterações do Método Newton até a convergência do método. 


\begin{table}[!htp]
\caption{Método de Newton para aproximação dos ângulos $\theta_1,~\theta_2,~\theta_3,~\theta_4$}
\label{tab:sys-nr}
\begin{tabular}{|c|c|c|c|c|c|}
\hline
\textbf{K} & \textbf{Ângulo $\theta_1$} & \textbf{Ângulo $\theta_2$} & \textbf{Ângulo $\theta_3$} & \textbf{Ângulo $\theta_4$} & \textbf{Norma da Matriz} \\ \hline
1          & 1                                    & 1                                    & 1                                    & 1                                    & 0,09247623274            \\ \hline
2          & 0,7203879844                         & 1,033592424                          & 1,154783643                          & 1,183712181                          & 0,027472394              \\ \hline
3          & 0,7123207365                         & 1,027283876                          & 1,14831363                           & 1,177085596                          & 0,00003998152952         \\ \hline
4          & 0,7123108904                         & 1,027273674                          & 1,148303632                          & 1,177075661                          & 0                        \\ \hline
\end{tabular}
\end{table}